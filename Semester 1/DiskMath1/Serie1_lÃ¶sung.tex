\documentclass[10pt,a4paper]{article}
\usepackage[utf8]{inputenc}
\usepackage{amsmath}
\usepackage{amsfonts}
\usepackage{amssymb}
\usepackage{graphicx}
\author{Gabriel Nadler}
\title{Diskrete Mathematik Übungsblatt 1}
\begin{document}
	\maketitle
	\paragraph{Aufgabe 1}
\begin{itemize}
	\item[a)]$\exists x \in\mathbb{N} \left(E(x)\right)$
	
	\item[b)] $\forall x \in\mathbb{N} \left(E(x)\right)$
	
	\item[c)] $\exists x \in\mathbb{N} \left[E(x) \land \forall y \in \mathbb{N} \left(E(y) \implies x = y \right) \right]$
	
	\item[d)] $\exists x,y,z \in \mathbb{N} \left(x \neq y \land x \neq z \land y \neq z \land E(x) \land E(y) \land E(z)\right)$
	
	\item[e)] $\neg$d)
\end{itemize}
\paragraph{Aufgabe 2}
\begin{itemize}
	\item[a)] $\forall x,y \in \mathbb{N} \left(PF(x,y) \implies Prim(x)\right)$
	
	\item[b)] $\forall y\left[y > 1 \implies \exists x \in \mathbb{N} \left(PF(x,y)\right)\right]$
	
	\item[c)] $\forall x \in \mathbb{N} \left[Prim(x) \implies \exists y \in \mathbb{N} \left(PF(x,y) \implies x = y \right)\right]$
	
	\item[d)] "Es gibt 1 Zahl mit genau einem Primfaktor die keine Primzahl ist"
	
	\item[e)] $\forall x \in \mathbb{N} \left[Prim(x) \implies \forall y \in \mathbb{N} \left(PF(y,x) \implies x = y \right)\right]$
\end{itemize}
\paragraph{Aufgabe 3}
\begin{itemize}
	\item[a)] Eine Zahl ist nie ein Primfaktor ihres Nachfolgers. (richtig)
	\item[b)] 2 ist die einzige Primzahl mit einer Primzahl als Nachfolger. (richtig)
	\item[c)] Es gibt 1 Zahl ohne Primfaktoren. (richtig: 1)
	\item[d)] Es gibt eine Zahl die kein Primfaktor ist (richtig: 1)

\end{itemize}
\end{document}